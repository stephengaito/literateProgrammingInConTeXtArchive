% A ConTeXt document [master document: literateProgs.tex]

\startchapter[title=Preamble]

Any description of almost any code artefact, begins with a preamble. The 
preamble sets the stage for the rest of the code. In our case, the 
respective preambles provide:
%
\startitemize
\item a machine readable description of the file,
\item a human readable copyright,
\item commands to use or include other artefacts
\item command to setup the correct internal state
\stopitemize

%\defineLitProgs[MkIVCode][option=context]
\defineLitProgs[LuaCode][option=lua]
\defineLitProgs[LuaTemplate][option=lua]

\startMkIVCode
%D \module
%D   [     file=t-literate-progs,
%D      version=2017.05.10,
%D        title=\CONTEXT\ User module,
%D     subtitle=Literate Programming in \ConTeXt,
%D       author=Stephen Gaito,
%D         date=\currentdate,
%D    copyright=PerceptiSys Ltd (Stephen Gaito),
%D        email=stephen@perceptisys.co.uk,
%D      license=MIT License]

%C Copyright (C) 2017 PerceptiSys Ltd (Stephen Gaito)
%C
%C Permission is hereby granted, free of charge, to any person obtaining a 
%C copy of this software and associated documentation files (the 
%C "Software"), to deal in the Software without restriction, including 
%C without limitation the rights to use, copy, modify, merge, publish, 
%C distribute, sublicense, and/or sell copies of the Software, and to 
%C permit persons to whom the Software is furnished to do so, subject to 
%C the following conditions: 
%C
%C The above copyright notice and this permission notice shall be included 
%C in all copies or substantial portions of the Software. 
%C
%C THE SOFTWARE IS PROVIDED "AS IS", WITHOUT WARRANTY OF ANY KIND, EXPRESS 
%C OR IMPLIED, INCLUDING BUT NOT LIMITED TO THE WARRANTIES OF 
%C MERCHANTABILITY, FITNESS FOR A PARTICULAR PURPOSE AND NONINFRINGEMENT. 
%C IN NO EVENT SHALL THE AUTHORS OR COPYRIGHT HOLDERS BE LIABLE FOR ANY 
%C CLAIM, DAMAGES OR OTHER LIABILITY, WHETHER IN AN ACTION OF CONTRACT, 
%C TORT OR OTHERWISE, ARISING FROM, OUT OF OR IN CONNECTION WITH THE 
%C SOFTWARE OR THE USE OR OTHER DEALINGS IN THE SOFTWARE. 

% begin info
%
% title   : Literate Programming in ConTeXt
% comment : Provides structured document and code generation
% status  : under development, mkiv only
%
% end info

\unprotect

\ctxloadluafile{t-literate-progs}
\ctxloadluafile{t-literate-progs-templates}
\stopMkIVCode

\startLuaCode
-- A Lua file

-- This is the lua code associated with t-literate-progs.mkiv

if not modules then modules = { } end
modules ['t-literate-progs'] = {
    version   = 1.000,
    comment   = "Literate Programming in ConTeXt - lua",
    author    = "PerceptiSys Ltd (Stephen Gaito)",
    copyright = "PerceptiSys Ltd (Stephen Gaito)",
    license   = "MIT License"
}

thirddata               = thirddata               or {}
thirddata.literateProgs = thirddata.literateProgs or {}

local litProgs   = thirddata.literateProgs
litProgs.code    = {}
local code       = litProgs.code
code.mkiv        = {}
code.lua         = {}
code.templates   = {}
code.lakefile    = {}
code.lineModulus = 50

local pp = require('pl/pretty')
local tInsert = table.insert
local tRemove = table.remove
local tConcat = table.concat
local tSort   = table.sort
local tUnpack = table.unpack
local sFmt    = string.format
local sMatch  = string.match
local mFloor  = math.floor
local toStr   = tostring
\stopLuaCode

\startLuaCode
local function markLuaCodeOrigin()
  code['LuaCode']      = code['LuaCode'] or { }
  local codeType       = code['LuaCode']
  local codeStream     = codeType.curCodeStream or 'default'
  codeType[codeStream] = codeType[codeStream] or { }
  codeStream           = codeType[codeStream]
  tInsert(codeStream,
    sFmt('-- from file: %s after line: %s',
      status.filename,
      toStr(
        mFloor(
          status.linenumber/code.lineModulus
        )*code.lineModulus
      )
    )
  )
end

litProgs.markLuaCodeOrigin = markLuaCodeOrigin
\stopLuaCode

\setLitProgsOriginMarker[LuaCode][markLuaCodeOrigin]

\section{Lua templates}

\startLuaTemplate
-- A Lua template file

-- t-literate-progs templates

if not modules then modules = { } end
modules ['t-literate-progs-templates'] = {
    version   = 1.000,
    comment   = "Literate Programming in ConTeXt - templates",
    author    = "PerceptiSys Ltd (Stephen Gaito)",
    copyright = "PerceptiSys Ltd (Stephen Gaito)",
    license   = "MIT License"
}

thirddata               = thirddata               or {}
thirddata.literateProgs = thirddata.literateProgs or {}

local litProgs          = thirddata.literateProgs
litProgs.templates      = {}
local templates         = litProgs.templates
templates.mkiv          = {}
templates.lua           = {}
templates.templates     = {}
templates.lakefile      = {}
templates.litProgsTable = {}

local table_insert = table.insert
local table_concat = table.concat

local addTemplate = litProgs.addTemplate
\stopLuaTemplate

\startLuaCode
local function markLuaTemplateOrigin()
  code['LuaTemplate']  = code['LuaTemplate'] or { }
  local codeType       = code['LuaTemplate']
  local codeStream     = codeType.curCodeStream or 'default'
  codeType[codeStream] = codeType[codeStream] or { }
  codeStream           = codeType[codeStream]
  tInsert(codeStream,
    sFmt('-- from file: %s after line: %s',
      status.filename,
      toStr(
        mFloor(
          status.linenumber/code.lineModulus
        )*code.lineModulus
      )
    )
  )
end

litProgs.markLuaTemplateOrigin = markLuaTemplateOrigin
\stopLuaCode

\setLitProgsOriginMarker[LuaTemplate][markLuaTemplateOrigin]


\stopchapter