% A ConTeXt document [master document: literateModules.tex]

\startchapter[title=Preamble]

Any description of almost any code artefact, begins with a preamble. The 
preamble sets the stage for the rest of the code. In our case, the 
respective preambles provide:
%
\startitemize
\item a machine readable description of the file,
\item a human readable copyright,
\item commands to use or include other artefacts
\item command to setup the correct internal state
\stopitemize

\startMkIVCode

% ConTeXt MkIV module
%D \module
%D   [     file=t-literate-modules,
%D      version=2017.05.10,
%D        title=\CONTEXT\ User module,
%D     subtitle=Literate \ConTeXt\ modules,
%D       author=Stephen Gaito,
%D         date=\currentdate,
%D    copyright=PerceptiSys Ltd (Stephen Gaito),
%D        email=stephen@perceptisys.co.uk,
%D      license=MIT License]

%C Copyright (C) 2017 PerceptiSys Ltd (Stephen Gaito)
%C
%C Permission is hereby granted, free of charge, to any person obtaining a 
%C copy of this software and associated documentation files (the 
%C "Software"), to deal in the Software without restriction, including 
%C without limitation the rights to use, copy, modify, merge, publish, 
%C distribute, sublicense, and/or sell copies of the Software, and to 
%C permit persons to whom the Software is furnished to do so, subject to 
%C the following conditions: 
%C
%C The above copyright notice and this permission notice shall be included 
%C in all copies or substantial portions of the Software. 
%C
%C THE SOFTWARE IS PROVIDED "AS IS", WITHOUT WARRANTY OF ANY KIND, EXPRESS 
%C OR IMPLIED, INCLUDING BUT NOT LIMITED TO THE WARRANTIES OF 
%C MERCHANTABILITY, FITNESS FOR A PARTICULAR PURPOSE AND NONINFRINGEMENT. 
%C IN NO EVENT SHALL THE AUTHORS OR COPYRIGHT HOLDERS BE LIABLE FOR ANY 
%C CLAIM, DAMAGES OR OTHER LIABILITY, WHETHER IN AN ACTION OF CONTRACT, 
%C TORT OR OTHERWISE, ARISING FROM, OUT OF OR IN CONNECTION WITH THE 
%C SOFTWARE OR THE USE OR OTHER DEALINGS IN THE SOFTWARE. 

% begin info
%
% title   : JoyLoL CoAlgebra definitions
% comment : Provides structured document and code generation
% status  : under development, mkiv only
%
% end info

%M \usemodule[literate-modules]

\usemodule[t-contests]

\unprotect

\ctxloadluafile{t-literate-modules}
\ctxloadluafile{t-literate-modules-templates}

\stopMkIVCode

\startLuaCode

-- A Lua file (the lua code associated with t-literate-modules.mkiv)

if not modules then modules = { } end modules ['t-literate-modules'] = {
    version   = 1.000,
    comment   = "Literate MkIV ConTeXt modules - lua",
    author    = "PerceptiSys Ltd (Stephen Gaito)",
    copyright = "PerceptiSys Ltd (Stephen Gaito)",
    license   = "MIT License"
}

thirddata                 = thirddata                 or {}
thirddata.literateModules = thirddata.literateModules or {}

local litMods  = thirddata.literateModules
litMods.code   = {}
local code     = litMods.code
code.mkiv      = {}
code.lua       = {}
code.templates = {}

local pp = require('pl/pretty')
local table_insert = table.insert
local table_concat = table.concat

\stopLuaCode

\section[title=The Lua rendering engine]

\startLuaCode

-- We need a simple Lua based template engine
-- Our template engine has been inspired by:
--   https://john.nachtimwald.com/2014/08/06/using-lua-as-a-templating-engine/
-- (via the minLua JoyLoL template engine)

function litMods.renderNextChunk(prevChunk, renderedText, curtemplate)
  local result = ""
  
  if prevChunk
    and type(prevChunk) == 'string'
    and 0 < #prevChunk then
    table_insert(renderedText, prevChunk)
  end
  
  if type(curTemplate) == 'string' and (0 < #curTemplate) then
    if curTemplate:find('{{') then
      local position  = 1
      local textChunk = curTemplate:match('^.*{{', position)
      if textChunk then 
        local textChunkLen = #textChunk
        textChunk = textChunk:sub(1, textChunkLen-2)
        if 0 < #textChunk then table_insert(renderedText, textChunk) end
        position = position + textChunkLen
      end
      
      local luaChunk = curTemplate:match('^.+}}', position)
      if luaChunk then
        local luaChunkLen = #luaChunk
        luaChunk = luaChunk:sub(1, luaChunkLen-2)
        position = position + luaChunkLen
        curTemplate = curTemplate:sub(position, #curTemplate)
        local newChunk = ""
        if not luaChunk:match('^%s*$') then
          -- consider using an PCall here....
          local luaFunc, errMessage = load(luaChunk)
          if luaFunc then
            newChunk = luaFunc(litMods)
          end
        end
        result = litMods.renderNextChunck(newChunk, renderedText, curTemplate)
      end
    else -- there is no '{{' in the template
      table_insert(renderedText, curTemplate)
      result = table_concat(renderedText)
    end
  else
    -- nothing to do...
    result = table_concat(renderedText)
  end
  return result
end

function litMods.render(aTemplate)
  return litMods.renderNextChunk("", { }, aTemplate)
end

-- Now we need the code that captures and creates a given code/file type 

local function renderFile(aFilePath, baseTemplate)
  local outFile = io.open(aFilePath, 'w')
  outFile:write(pp.write(litMods))
  local renderedBaseTemplate = litMods.renderNextChunk("", {}, baseTemplate)
  outFile:write('\n--------------\n')
  outFile:write(renderedBaseTemplate)
  outFile:close()
end

\stopLuaCode

\stopchapter